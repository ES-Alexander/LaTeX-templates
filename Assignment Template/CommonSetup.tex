% ------------- GENERAL -------------- %
\usepackage{amsfonts} % fonts package
% math, symbols, and theorems packages (esint -> oiint, mathrsfs -> \mathscr{L})
\usepackage{amsmath, amsthm, amssymb, esint, mathrsfs} 
\usepackage{enumerate} % Allows for roman numbering of lists
% colour package (colour!num -> colour @ num% opacity)
\usepackage{xcolor} 
\usepackage{listings} % nicely formatted code

% -------------- HEADER --------------- %
\usepackage{fancyhdr} % creates header
\pagestyle{fancy}
\fancyhf{}
\lhead{\leftHeader} % left header
\rhead{\rightHeader} % right header
% page number bottom right (instead of default middle)
\rfoot{\thepage} 

%-------------------FIGURES ---------------------%
\usepackage{graphicx, float, caption}
% package enabling subfigures
\usepackage{subcaption}

% \fig{scale 1=100%}{filename}{caption \label{labelname}}  (label optional)
\newcommand{\fig}[3]{\begin{figure}[H] 
    \centering
    \includegraphics[scale=#1]{#2}
    \caption{#3}
\end{figure}}

% Insert figures with a frame
\newcommand{\framefig}[3]{\begin{figure}[H] 
    \centering
    \frame{\includegraphics[scale=#1]{#2}}
    \caption{#3}
\end{figure}}

% Insert sub-figures (within a figure)
%\subfig{alignment}{fig-width}{img-scale/height/width}{filename}{caption}
% alignment = t(top) / b(bottom)
% width e.g. = 0.5\textwidth
% scale e.g. = scale=0.5 / height=5cm / width=3in
% filename e.g. = image / image.jpg / etc
% caption e.g. = Descriptive subfigure caption
\newcommand{\subfig}[5]{
    \begin{subfigure}[#1]{#2}
        \centering
        \includegraphics[#3]{#4}
        \caption{#5}
    \end{subfigure}
}

%-------------------COMMENTS---------------------%
% Insert multi-line comments with \comment{<text>}
\newcommand{\comment}[1]{} 
% Insert displayed multi-line comments using \dcom{<text>}
\newcommand{\dcom}[1]{\textcolor{red}{\textit{#1}}}

%-------------CODE FORMATTING---------------%
% Matlab
\usepackage[numbered,framed]{matlab-prettifier} % MATLAB nice code

% Python
% Default fixed font does not support bold face
\DeclareFixedFont{\ttb}{T1}{txtt}{bx}{n}{10} % for bold
\DeclareFixedFont{\ttm}{T1}{txtt}{m}{n}{10}  % for normal
% Custom colors
\definecolor{deepblue}{rgb}{0,0,0.5}
\definecolor{deepred}{rgb}{0.6,0,0}
\definecolor{deepgreen}{rgb}{0,0.5,0}
\definecolor{lightgrey}{rgb}{0.7,0.7,0.7}
\definecolor{black}{rgb}{0,0,0}
\newcommand{\pythonstyle}{language=Python,
            numbers=left, numberstyle=\color{lightgrey}\small,
            numbersep=8pt,
            basicstyle=\ttm, otherkeywords={self},
            keywordstyle=\ttb\color{deepblue},
            emph={MyClass,__init__},          % Custom highlighting
            emphstyle=\ttb\color{deepred},    % Custom highlighting style
            stringstyle=\color{deepgreen},
            commentstyle=\color{deepred},
            frame=single,
            showstringspaces=false,
            breaklines=true} % Python nice code format
% Use as \lstinputlisting[\python]{code/filename.py}

\newcounter{codeCounter}

\newcommand{\codeCaption}[1]{
    \centering{ % braces prevent centering of document outside of caption
    \stepcounter{codeCounter}
    \vspace{\topsep}
    Listing \arabic{codeCounter}: {#1}}
}

\newcommand{\python}[2]{
    \lstinputlisting[\pythonstyle]{#1}
    \codeCaption{#2}
}

\newcommand{\matlab}[2]{
    \lstset{style=Matlab-editor}
    \lstinputlisting{#1}
    \codeCaption{#2}
}

%-----------------REFERENCING------------------%
% Reference figures:
\newcommand{\figref}[1]{Figure \ref{#1}}
% Reference subfigures:
\newcommand{\sfigref}[1]{Figure \ref{#1})}
% Reference tables:
\newcommand{\tabref}[1]{Table \ref{#1}}
% Reference code:
\newcommand{\coderef}[1]{Listing \ref{#1}}
% Reference sections:
\newcommand{\secref}[1]{Section \ref{#1}}
% Reference appendices:
\newcommand{\appref}[1]{Appendix \ref{#1}}

%--------------MATHS NOTATIONS---------------%
% Insert units with \unit{<unit>} (e.g. \unit{m}^3)
\newcommand{\unit}[1]{\ \text{#1}}
\newcommand{\nsunit}[1]{\text{#1}}

% Insert *10^n with \E{n} (e.g. \E{-5})
\newcommand{\E}[1]{\times 10^{#1}}
\newcommand{\tE}[1]{$\E{#1}$}

% Command for math formatted differentials (use \D var)
\newcommand{\D}{\ \mathrm{d}} 

% Command for degrees symbol
\renewcommand{\deg}[1]{{#1}^\circ}

% Command for short bolded vector notation
\newcommand{\ve}[1]{\mathbf{#1}}

% Commands for left/right surrounding brackets
\newcommand{\lr}[1]{\left( #1 \right)}
\newcommand{\lrs}[1]{\left[ #1 \right]}
\newcommand{\lra}[1]{\left | #1 \right |}

% Commands for integrals
\newcommand{\eint}[2]{\int #1 \D #2}
\newcommand{\eiint}[3]{\iint #1 \D #2 \D #3}
\newcommand{\dint}[4]{\int_{#1}^{#2} #3 \D #4} % definite integral
\newcommand{\diint}[7]{\dint{#1}{#2}{\dint{#3}{#4}{#5}{#6}}{#7}}

% Command for partial derivatives
\newcommand{\del}[2]{\dfrac{\partial #1}{\partial #2}}

% Command for full derivatives (nth order)
\newcommand{\der}[3][]{\dfrac{\D ^{#1} #2}{\D {#3}^{#1}}}

% Command for common redefinitions (brackets included)
\renewcommand{\cos}[2][]{\text{cos}^{#1}\!\lr{#2}} % optional power
\renewcommand{\sin}[2][]{\text{sin}^{#1}\!\lr{#2}} % optional power
\renewcommand{\tan}[2][]{\text{tan}^{#1}\!\lr{#2}} % optional power
\renewcommand{\ln}[1]{\text{ln}\!\lr{#1}}
\renewcommand{\log}[2][]{\text{log}_{#1}\!\lr{#2}} % optional base

% Command for quadratic formula
\newcommand{\qf}[3]{\dfrac{-(#2)\pm\sqrt{(#2)^2-4\cdot #1 \cdot #3}}{2\cdot #1}}

% Command for 2x2 eigenvalues working
\newcommand{\evals}[5]{0 &= \det(#1 - \lambda I)\\
				&= \left| \begin{pmatrix} #2 & #3\\ #4 & #5 \end{pmatrix} - 
					\begin{pmatrix} \lambda & 0\\ 0 & \lambda \end{pmatrix} \right|\\
				&= \begin{vmatrix}
						#2 - \lambda & #3\\
						#4 & #5 - \lambda
					\end{vmatrix}\\
				&= (#2-\lambda)(#5 - \lambda) - (#3)(#4)}

% Command for 2x2 eigenvectors working
\newcommand{\evecs}[8]{#1 #2 &= \lambda #2\\
				(#1 - \lambda I) #2 &= \ve{0}\\
				\begin{pmatrix}
						#5 - \lambda & #6\\
						#7 & #8 - \lambda
					\end{pmatrix} \begin{pmatrix}
						#3 \\ #4
					\end{pmatrix} &= \begin{pmatrix}
						0 \\ 0
					\end{pmatrix}}