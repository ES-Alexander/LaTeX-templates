% ------------- GENERAL -------------- %
\usepackage[utf8]{inputenc} % allows for online editing
\usepackage{amsfonts} % fonts package
% math, symbols, and theorems packages (esint -> oiint, mathrsfs -> \mathscr{L})
\usepackage{amsmath, amsthm, amssymb, esint, mathrsfs} 
\usepackage{mathtools} % Provides \Aboxed{...}, boxed align equations.
\usepackage{enumitem} % Allows for roman numbering of lists
% colour package (colour!num -> colour @ num% opacity)
\usepackage{xcolor} 
\usepackage{listings} % nicely formatted code

% -------------- HEADER --------------- %
\usepackage{fancyhdr} % creates header
\pagestyle{fancy}
\fancyhf{}
\lhead{\leftHeader} % left header
\rhead{\rightHeader} % right header
% page number bottom right (instead of default middle)
\rfoot{\thepage}

% ------------ PAGE SIZES ------------ %
% Long side of A4 paper
\defineLength{\aflong}{29.7cm}
% Short side of A4 paper
\defineLength{\afshort}{21cm}
% Long side of A3 paper
\defineLength{\atlong}{42cm}
% Short side of A3 paper
\defineLength{\atshort}{29.7cm}

% Text width can be determined using (pagewidth - 2\totalmargin)

% Set page size with:
% \eject \pdfpagewidth=<width> \pdfpageheight=<height>

% Set a page to a new size (\setpagesize{width}{height})
\newcommand{\setpagesize}[2]{
  \eject 
	% set height and width
	\pdfpagewidth = #1 \pdfpageheight = #2
	% adjust horizontal and vertical offset w.r.t. a4 portrait basis
	\newgeometry{
		top = \topMargin, 
		bottom = \bottomMargin - (#2 - \aflong),
		left = \leftMargin,
		right = \rightMargin - (#1 - \afshort)
	}
	% refresh header and footer to suit new page size
	\fancyheadoffset[R]{0cm}  \fancyfootoffset[R]{0cm}
}

% Sets page to A4 portrait
\newcommand{\setafport}{\setpagesize{\afshort}{\aflong}}

% Sets page to A3 landscape
\newcommand{\setatland}{\setpagesize{\atlong}{\atshort}}

% Sets page to A4 landscape
\newcommand{\setafland}{\setpagesize{\aflong}{\afshort}}

% Sets page to A3 portrait
\newcommand{\setatport}{\setpagesize{\atshort}{\atlong}}

%------------------- LABELS ---------------------%
% Label with a prefix
% \plabel[prefix]{label}
\newcommand{\plabel}[2][]{\label{#1: #2}}

% Label for an equation
% \eqlabel{label}
\newcommand{\eqlabel}[1]{\plabel[eq]{#1}}

%------------------- FIGURES --------------------%
\usepackage{graphicx, float, caption}
% package enabling subfigures
\usepackage{subcaption}

% Insert a centered figure
% \fig[alignment]{scale}{filename}{caption}
%   *alignment: H by default for 'exactly in place' - can also be:
%     h (approximately in place)
%     b (at the bottom)
%     t (at the top)
%     p (on a page for figures/tables)
%     (some others possible if necessary)
%   scale: the scale of the inputted image (1 = real size)
%   filename: don't include the extension (e.g. use photo instead of photo.png)
%             unless multiple files called photo exist with different extensions
%   caption: figure caption
%
%   auto-labelling labels by the filename, reference with \figref{filename} or
%     \reffig{filename} for just the number
\newcommand{\fig}[4][H]{
  \begin{figure}[#1]
	  \centering
	  \includegraphics[scale=#2]{#3}
	  \caption{#4 \plabel[fig]{#3}}
  \end{figure}
}

% Width-sized figure \figw[alignment]{width}{filename}{caption}
%  e.g. \figw{0.9\textwidth}{my_img}{I am scaled to 9/10 of the textwidth}
\newcommand{\figw}[4][H]{
  \begin{figure}[#1]
	  \centering
	  \includegraphics[width=#2]{#3}
	  \caption{#4 \plabel[fig]{#3}}
  \end{figure}
}

% Insert a centered figure with a frame
% \framefig[alignment]{scale}{filename}{caption}
\newcommand{\framefig}[4][H]{
  \begin{figure}[#1] 
	  \centering
	  \frame{\includegraphics[scale=#2]{#3}}
	  \caption{#4 \plabel[fig]{#3}}
  \end{figure}
}

% Insert multiple subfigures
% \multifig[alignment]{\subfigs}{caption}
%
%   auto-labelling labels by the caption, reference with \figref{caption} or
%     \reffig{caption} for just the number
\newcommand{\multifig}[3][H]{
  \begin{figure}[#1]
    \centering
    #2
    \caption{#3 \plabel[fig]{#3}}
  \end{figure}
}

% Insert sub-figures (within a multifig)
% \subfig[img-scale/height/width]{alignment}{fig-width}{filename}{caption}
%   alignment: applies to within the figure (normally want t (top))
%
%   auto-labelling labels by the filename, reference with \sfigref{filename} or
%     \reffig[s]{filename} for just the number
\newcommand{\subfig}[5][width=\textwidth]{
  \begin{subfigure}[#2]{#3}
    \centering
    \includegraphics[#1]{#4}
    \caption{#5 \plabel[sfig]{#4}}
  \end{subfigure}
}

%--------------------TABLES----------------------%
\usepackage{array} % extended column definitions
% setup
\newcolumntype{^}{>{\global\let\currentrowstyle\relax}}
\newcolumntype{;}{>{\currentrowstyle}}
\newcommand{\rowstyle}[1]{\gdef\currentrowstyle{#1}%
  #1\ignorespaces
}
\newcolumntype{b}{>{\bfseries}} % bold column
\newcolumntype{i}{>{\itshape}}  % italicised column

% useful shortcuts
\newcommand{\bfrow}{\rowstyle{\bfseries}}
\newcommand{\itrow}{\rowstyle{\itshape}}
\newcommand{\headingrow}{\hline\bfrow}
\newcommand{\row}{\\ \hline} % end rows with this

% Create a table
% \easytable[alignment]{caption}{columns}{contents}
%   *alignment: H by default for 'exactly in place' - can also be:
%     h (approximately in place)
%     b (at the bottom)
%     t (at the top)
%     p (on a page for figures/tables)
%     (some others possible if necessary)
%   caption: table caption
%   columns: column distribution, types, and separation 
%     To allow for full row styling, first column specifier should be preceded
%       by ^ and following column specifiers should each be preceded by ;
%       e.g. |^c|;c|;p{0.5\linewidth}|
%     p{width} = justified, with width as specified 
%       (requires units (2cm, 5em, 12pt), or relative measure (n\linewidth))
%     c = centered, with width auto-scaled
%     l = left aligned
%     r = right aligned
%     | = vertical border line (can be repeated, e.g. |^c||;l;r)
%     spaces are ignored
%     To style a whole column, precede its alignment specifier with:
%       b (bold)
%       i (italicised)
%       >{style} (styled as specified)
%   contents: table contents
%     e.g. 
%       \headingrow Heading 1 & Heading 2 & Heading 3 \row
%       Side Heading & Text & Text \row
%     To style a single cell, use 
%       \textbf{text}      (bold)
%       \textit{text}      (italicised)
%       \normalfont{text}  (ignore column/row styling)
%     To style a row, precede it with:
%       \boldrow          (bold)
%       \italicrow        (italicised)
%       \rowstyle{style}  (styled as specified)
\newcommand{\easytable}[4][H]{
  \begin{table}[#1]
    \footnotesize % sets text size to that of footnotes, within table
    \centering % centers table on screen
    \caption{#2}
    \plabel[tab]{#2}
    \begin{tabular}{#3} 
      #4
    \end{tabular}
  \end{table}
}

%-------------------COMMENTS---------------------%
% Insert multi-line comments with \comment{<text>}
\newcommand{\comment}[1]{} 
% Insert displayed multi-line comments using \dcom{<text>}
\newcommand{\dcom}[1]{\textcolor{red}{\textit{#1}}}

%------------------GENERAL TEXT------------------%
% removes indent at beginning of all paragraph
\setlength\parindent{0pt} 
% removes line spacing between list items.
\setlist{noitemsep} 

%-------------CODE FORMATTING---------------%
%-------------CODE FORMATTING---------------%
% Matlab
\usepackage[numbered,framed]{matlab-prettifier} % MATLAB nice code

% Python
% Default fixed font does not support bold face
\DeclareFixedFont{\ttb}{T1}{txtt}{bx}{n}{10} % for bold
\DeclareFixedFont{\ttm}{T1}{txtt}{m}{n}{10}  % for normal
% Custom colors
\definecolor{deepblue}{rgb}{0,0,0.5}
\definecolor{deepred}{rgb}{0.6,0,0}
\definecolor{deepgreen}{rgb}{0,0.5,0}
\definecolor{lightgrey}{rgb}{0.7,0.7,0.7}
\definecolor{orange}{rgb}{0.9,0.5,0.2}
\definecolor{blue}{rgb}{0,0,0.7}
\definecolor{lightblue}{rgb}{0.24,0.65,1}

\newcounter{codeCounter}

% use as \codeCaption[label]{caption}
%   if no label is provided, the caption is used as the label
\newcommand{\codeCaption}[2][]{
  \refstepcounter{codeCounter}
	\begin{center}
		Listing \arabic{codeCounter}: {#2
			\ifx&#1&%
			  \plabel[code]{#2} % optional arg is empty, use caption as label
			\else
			  \plabel[code]{#1} % optional arg available, use it as label
			\fi
		}
	\end{center}
}

\lstdefinestyle{python-IDLE}{
  language=Python, 
  numbers=left, 
  numberstyle=\color{lightgrey}\small, 
  numbersep=8pt, 
  basicstyle=\ttm,
  basewidth={.012\textwidth}, % Inter-letter spacing
  otherkeywords={self,cls},           % Custom keywords
  keywordstyle=\ttb\color{deepblue},  % Keywords style
  emph={__init__,__call__,__eq__,__str__,__repr__}, % Custom highlighting
  emphstyle=\ttb\color{deepred},                    % Highlighting style
  stringstyle=\color{deepgreen}, 
  commentstyle=\color{deepred},
  frame=single, 
  showstringspaces=false, 
  breaklines=true
}

\lstdefinestyle{C-MPLAB-X}{
  language=C,
  numbers=left,
  numberstyle=\color{lightgrey}\small, 
  numbersep=8pt, 
  basicstyle=\ttm,
  basewidth={.0125\textwidth}, % Inter-letter spacing
  otherkeywords={uint8_t,uint16_t,int16_t,bool,\#define}, % Custom keywords
  keywordstyle=\ttb\color{blue},  % Keywords style
  emph={__delay_ms, defined, true, false}, % Custom highlighting
  emphstyle=\ttb\color{lightblue},  % Highlighting style
  stringstyle=\color{orange}, 
  commentstyle=\color{deepgreen},
  frame=single, 
  showstringspaces=false, 
  breaklines=true
}

\lstdefinestyle{pseudocode}{
  language=Java,
  numbers=left,
  numberstyle=\color{lightgrey}\small, 
  numbersep=8pt, 
  basicstyle=\ttm,
  basewidth={.0125\textwidth}, % Inter-letter spacing
  keywordstyle=\ttb\color{blue},  % Keywords style
  otherkeywords={Algorithm, Input, Output, then, <, >, -, =, +, *, true, false},
  emph={O, ops}, % Custom highlighting
  emphstyle=\ttb\color{red},  % Highlighting style
  stringstyle=\color{orange}, 
  commentstyle=\color{deepgreen},
  frame=single, 
  showstringspaces=false, 
  breaklines=true
}

\lstset{basicstyle=\ttfamily, basewidth={0.0125\textwidth}}

% \intextCode[options]{caption}{code}
\newcommand{\intextCode}[3][]{
  \codeCaption{#2}
  \begin{lstlisting}[#1]
#3
  \end{lstlisting}
}

% \inputCode[options]{filename}{caption}
\newcommand{\inputCode}[3][]{
  \codeCaption[#2]{#3}
  \lstinputlisting[#1]{#2}
}

% use as \python[options]{filename}{caption}
\newcommand{\python}[3][]{
  \inputCode[style=python-IDLE, #1]{#2}{#3}
}

% use as \matlab[options]{filename}{caption}
\newcommand{\matlab}[3][]{
  \inputCode[style=Matlab-editor, #1]{#2}{#3}
}

% use as \C[options]{filename}{caption}
\newcommand{\C}[3][]{
  \inputCode[style=C-MPLAB-X, #1]{#2}{#3}
}

%-----------------REFERENCING------------------%
% Just the number
\newcommand{\reftab}[1]{\ref{tab: #1}}     % tables
\newcommand{\reffig}[2][]{\ref{#1fig: #2}} % figures/subfigures
\newcommand{\refcode}[1]{\ref{code: #1}}   % code/listings
\renewcommand{\eqref}[1]{(\ref{eq: #1})}   % equations

% Number + relevant intro
\newcommand{\figref}[1]{Figure \reffig{#1}}       % figures
\newcommand{\sfigref}[1]{Figure \reffig[s]{#1})}  % subfigures
\newcommand{\tabref}[1]{Table \reftab{#1}}        % tables
\newcommand{\coderef}[1]{Listing \refcode{#1}}    % code/listings

%--------------MATHS NOTATIONS---------------%
% Insert units with \unit[power]{<unit>} (e.g. \unit[3]{m} -> m^3)
\newcommand{\unit}[2][]{\,\text{#2}^{#1}}
\newcommand{\nsunit}[2][]{\text{#2}^{#1}}

% Insert *10^n with \E{n} (e.g. \E{-5})
\newcommand{\E}[1]{\times 10^{#1}}
\newcommand{\tE}[1]{$\E{#1}$}

% Command for math formatted differentials (use \D var)
\newcommand{\D}{\,\mathrm{d}} 

% Command for degrees symbol
\renewcommand{\deg}[1][]{{}^\circ\nsunit{#1}}

% Command for short bolded vector notation
\newcommand{\ve}[1]{\mathbf{#1}}

% Commands for left/right surrounding brackets
\newcommand{\lr}[1]{\left( #1 \right)}
\newcommand{\lrs}[1]{\left[ #1 \right]}
\newcommand{\lra}[1]{\left | #1 \right |}

% Commands for integrals
\newcommand{\eint}[2]{\int #1 \D #2}
\newcommand{\eiint}[3]{\iint #1 \D #2 \D #3}
\newcommand{\dint}[4]{\int_{#1}^{#2} #3 \D #4} % definite integral
\newcommand{\diint}[7]{\dint{#1}{#2}{\dint{#3}{#4}{#5}{#6}}{#7}}

% Command for partial derivatives
\newcommand{\del}[3][]{\dfrac{\partial^{#1} #2}{\partial #3^{#1}}}

% Command for full derivatives (nth order)
\newcommand{\der}[3][]{\dfrac{\D ^{#1} #2}{\D {#3}^{#1}}}

% Command for common redefinitions (brackets included)
\renewcommand{\cos}[2][]{\mathrm{cos}^{#1}\!\lr{#2}} % optional power
\renewcommand{\sin}[2][]{\mathrm{sin}^{#1}\!\lr{#2}} % optional power
\renewcommand{\tan}[2][]{\mathrm{tan}^{#1}\!\lr{#2}} % optional power
\renewcommand{\arccos}[1]{\mathrm{arccos}\!\lr{#1}}
\renewcommand{\arcsin}[1]{\mathrm{arcsin}\!\lr{#1}}
\renewcommand{\arctan}[1]{\mathrm{arctan}\!\lr{#1}}
\renewcommand{\ln}[1]{\mathrm{ln}\!\lr{#1}}
\renewcommand{\log}[2][]{\mathrm{log}_{#1}\!\lr{#2}} % optional base
\renewcommand{\exp}[1]{\mathrm{exp}\!\lr{#1}}
\newcommand{\limit}[2]{\lim\limits_{{#1}\rightarrow{#2}}}

% Command for quadratic formula
\newcommand{\qf}[3]{\dfrac{-(#2)\pm\sqrt{(#2)^2-4\cdot #1 \cdot #3}}{2\cdot #1}}
