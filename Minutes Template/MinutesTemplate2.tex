\documentclass[11pt, a4paper]{article}

% ---------- DOCUMENT INFORMATION -----------%

% Meeting number/ID
\newcommand{\meetingID}{X}

% Meeting Descriptor
\newcommand{\meetingDescriptor}{XYZ}

% Time and Location
\newcommand{\setDate}{XX/XX/20XX}
\newcommand{\setTime}{XX:XX}
\newcommand{\place}	{XX - XX}

% People Present
\newcommand{\peoplePresent}{
	First1 Last1\\
	First2 Last2
}

% Includes the relevant setup for a report.
% Access this file to change heading colours and other setup features
% Margin (comment out the undesired option)

% Set all using totalmargin
\deflength{\totalMargin}{2.5cm}

\deflength{\topMargin}{\totalMargin}
\deflength{\bottomMargin}{\totalMargin}
\deflength{\leftMargin}{\totalMargin}
\deflength{\rightMargin}{\totalMargin}

% Set each individually
%\deflength{\topmargin}{2.5cm}
%\deflength{\bottommargin}{2.5cm}
%\deflength{\leftmargin}{2.5cm}
%\deflength{\rightmargin}{2.5cm}


% Combined Meeting Information
\newcommand{\meetingInfo}{
	\begin{minipage}{0.5\textwidth}
		\large \textbf{\textit{Date:}} \setDate\\
		\large \textbf{\textit{Time:}} \setTime\\
		\large \textbf{\textit{Location:}} \place
	\end{minipage}
	~
	\begin{minipage}{0.5\textwidth}
		\large \textbf{\textit{Present:}}\\[-0.7cm]
		\begin{itemize}
			\item[] \peoplePresent
		\end{itemize}
	\end{minipage}\\[0.5cm]
	\HRule
}
		
% Header
\newcommand{\rightHeader}{Meeting \meetingID}
\newcommand{\leftHeader}{\meetingDescriptor}

% Section Colours (RGB)
\newcommand{\sectionColour}{0, 51, 204}
\newcommand{\subsectionColour}{31, 73, 125}
\newcommand{\subsubsectionColour}{79, 129, 189}

% Table Header Colour (RGB)
\newcommand{\tableHeadColour}{219,229,241}

% allows for online editing and stuff
\usepackage[utf8]{inputenc} 

% ------------- GENERAL -------------- %
\usepackage{amsfonts} % fonts package
% math, symbols, and theorems packages (esint -> oiint, mathrsfs -> \mathscr{L})
\usepackage{amsmath, amsthm, amssymb, esint, mathrsfs} 
\usepackage{mathtools} % Provides \Aboxed{...}, boxed align equations.
\usepackage{enumerate} % Allows for roman numbering of lists
% colour package (colour!num -> colour @ num% opacity)
\usepackage{xcolor} 
\usepackage{listings} % nicely formatted code

% -------------- HEADER --------------- %
\usepackage{fancyhdr} % creates header
\pagestyle{fancy}
\fancyhf{}
\lhead{\leftHeader} % left header
\rhead{\rightHeader} % right header
% page number bottom right (instead of default middle)
\rfoot{\thepage} 

% ------------ PAGE SIZES ------------ %
% Long side of A4 paper
\deflength{\aflong}{29.7cm}
% Short side of A4 paper
\deflength{\afshort}{21cm}
% Long side of A3 paper
\deflength{\atlong}{42cm}
% Short side of A3 paper
\deflength{\atshort}{29.7cm}

% Text width can be determined using (pagewidth - 2\totalmargin)

% Set page size with:
% \eject \pdfpagewidth=<width> \pdfpageheight=<height>

% Set a page to a new size (\setpagesize{width}{height})
\newcommand{\setpagesize}[2]{
		\eject 
		% set height and width
		\pdfpagewidth = #1 \pdfpageheight = #2
		% adjust horizontal and vertical offset w.r.t. a4 portrait basis
		\newgeometry{
			top = \topMargin, 
			bottom = \bottomMargin - (#2 - \aflong),
			left = \leftMargin,
			right = \rightMargin - (#1 - \afshort)
		}
		% refresh header and footer to suit new page size
		\fancyheadoffset[R]{0cm}  \fancyfootoffset[R]{0cm}
}

% Sets page to A4 portrait
\newcommand{\setafport}{\setpagesize{\afshort}{\aflong}}

% Sets page to A3 landscape
\newcommand{\setatland}{\setpagesize{\atlong}{\atshort}}

% Sets page to A4 landscape
\newcommand{\setafland}{\setpagesize{\aflong}{\afshort}}

% Sets page to A3 portrait
\newcommand{\setatport}{\setpagesize{\atshort}{\atlong}} 

%-------------------FIGURES ---------------------%
\usepackage{graphicx, float, caption}
% package enabling subfigures
\usepackage{subcaption}

% \fig{scale 1=100%}{filename}{caption \label{labelname}}  (label optional)
\newcommand{\fig}[3]{\begin{figure}[H] 
    \centering
    \includegraphics[scale=#1]{#2}
    \caption{#3}
\end{figure}}

% Insert figures with a frame
\newcommand{\framefig}[3]{\begin{figure}[H] 
    \centering
    \frame{\includegraphics[scale=#1]{#2}}
    \caption{#3}
\end{figure}}

% Insert sub-figures (within a figure)
%\subfig{alignment}{fig-width}{img-scale/height/width}{filename}{caption}
% alignment = t(top) / b(bottom)
% width e.g. = 0.5\textwidth
% scale e.g. = scale=0.5 / height=5cm / width=3in
% filename e.g. = image / image.jpg / etc
% caption e.g. = Descriptive subfigure caption
\newcommand{\subfig}[5]{
    \begin{subfigure}[#1]{#2}
        \centering
        \includegraphics[#3]{#4}
        \caption{#5}
    \end{subfigure}
}

%-------------------COMMENTS---------------------%
% Insert multi-line comments with \comment{<text>}
\newcommand{\comment}[1]{} 
% Insert displayed multi-line comments using \dcom{<text>}
\newcommand{\dcom}[1]{\textcolor{red}{\textit{#1}}}

%-------------CODE FORMATTING---------------%
% Matlab
\usepackage[numbered,framed]{matlab-prettifier} % MATLAB nice code

% Python
% Default fixed font does not support bold face
\DeclareFixedFont{\ttb}{T1}{txtt}{bx}{n}{10} % for bold
\DeclareFixedFont{\ttm}{T1}{txtt}{m}{n}{10}  % for normal
% Custom colors
\definecolor{deepblue}{rgb}{0,0,0.5}
\definecolor{deepred}{rgb}{0.6,0,0}
\definecolor{deepgreen}{rgb}{0,0.5,0}
\definecolor{lightgrey}{rgb}{0.7,0.7,0.7}
\definecolor{black}{rgb}{0,0,0}
\newcommand{\pythonstyle}{language=Python,
            numbers=left, numberstyle=\color{lightgrey}\small,
            numbersep=8pt,
            basicstyle=\ttm, otherkeywords={self},
            keywordstyle=\ttb\color{deepblue},
            emph={MyClass,__init__},          % Custom highlighting
            emphstyle=\ttb\color{deepred},    % Custom highlighting style
            stringstyle=\color{deepgreen},
            commentstyle=\color{deepred},
            frame=single,
            showstringspaces=false,
            breaklines=true} % Python nice code format
% Use as \lstinputlisting[\python]{code/filename.py}

\newcounter{codeCounter}

\newcommand{\codeCaption}[1]{
    \refstepcounter{codeCounter}
   	\vspace{-\topsep}
	\begin{center}
		Listing \arabic{codeCounter}: {#1}
	\end{center}
}

\newcommand{\python}[2]{
	\vspace{\topsep}
    \lstinputlisting[\pythonstyle]{#1}
    \codeCaption{#2}
}

\newcommand{\matlab}[2]{
	\vspace{\topsep}
    \lstset{style=Matlab-editor}
    \lstinputlisting{#1}
    \codeCaption{#2}
}

%-----------------REFERENCING------------------%
% Reference figures:
\newcommand{\figref}[1]{Figure \ref{#1}}
% Reference subfigures:
\newcommand{\sfigref}[1]{Figure \ref{#1})}
% Reference tables:
\newcommand{\tabref}[1]{Table \ref{#1}}
% Reference code:
\newcommand{\coderef}[1]{Listing \ref{#1}}
% Reference sections:
\newcommand{\secref}[1]{Section \ref{#1}}
% Reference appendices:
\newcommand{\appref}[1]{Appendix \ref{#1}}

%--------------MATHS NOTATIONS---------------%
% Insert units with \unit{<unit>} (e.g. \unit{m}^3)
\newcommand{\unit}[1]{\ \text{#1}}
\newcommand{\nsunit}[1]{\text{#1}}

% Insert *10^n with \E{n} (e.g. \E{-5})
\newcommand{\E}[1]{\times 10^{#1}}
\newcommand{\tE}[1]{$\E{#1}$}

% Command for math formatted differentials (use \D var)
\newcommand{\D}{\ \mathrm{d}} 

% Command for degrees symbol
\renewcommand{\deg}[1]{{#1}^\circ}

% Command for short bolded vector notation
\newcommand{\ve}[1]{\mathbf{#1}}

% Commands for left/right surrounding brackets
\newcommand{\lr}[1]{\left( #1 \right)}
\newcommand{\lrs}[1]{\left[ #1 \right]}
\newcommand{\lra}[1]{\left | #1 \right |}

% Commands for integrals
\newcommand{\eint}[2]{\int #1 \D #2}
\newcommand{\eiint}[3]{\iint #1 \D #2 \D #3}
\newcommand{\dint}[4]{\int_{#1}^{#2} #3 \D #4} % definite integral
\newcommand{\diint}[7]{\dint{#1}{#2}{\dint{#3}{#4}{#5}{#6}}{#7}}

% Command for partial derivatives
\newcommand{\del}[2]{\dfrac{\partial #1}{\partial #2}}

% Command for full derivatives (nth order)
\newcommand{\der}[3][]{\dfrac{\D ^{#1} #2}{\D {#3}^{#1}}}

% Command for common redefinitions (brackets included)
\renewcommand{\cos}[2][]{\text{cos}^{#1}\!\lr{#2}} % optional power
\renewcommand{\sin}[2][]{\text{sin}^{#1}\!\lr{#2}} % optional power
\renewcommand{\tan}[2][]{\text{tan}^{#1}\!\lr{#2}} % optional power
\renewcommand{\ln}[1]{\text{ln}\!\lr{#1}}
\renewcommand{\log}[2][]{\text{log}_{#1}\!\lr{#2}} % optional base
\newcommand{\limit}[2]{\lim\limits_{{#1}\rightarrow{#2}}}

% Command for quadratic formula
\newcommand{\qf}[3]{\dfrac{-(#2)\pm\sqrt{(#2)^2-4\cdot #1 \cdot #3}}{2\cdot #1}}

% Command for 2x2 eigenvalues working
\newcommand{\evals}[5]{0 &= \det(#1 - \lambda I)\\
				&= \left| \begin{pmatrix} #2 & #3\\ #4 & #5 \end{pmatrix} - 
					\begin{pmatrix} \lambda & 0\\ 0 & \lambda \end{pmatrix} \right|\\
				&= \begin{vmatrix}
						#2 - \lambda & #3\\
						#4 & #5 - \lambda
					\end{vmatrix}\\
				&= (#2-\lambda)(#5 - \lambda) - (#3)(#4)}

% Command for 2x2 eigenvectors working
\newcommand{\evecs}[8]{#1 #2 &= \lambda #2\\
				(#1 - \lambda I) #2 &= \ve{0}\\
				\begin{pmatrix}
						#5 - \lambda & #6\\
						#7 & #8 - \lambda
					\end{pmatrix} \begin{pmatrix}
						#3 \\ #4
					\end{pmatrix} &= \begin{pmatrix}
						0 \\ 0
					\end{pmatrix}}

% sets the margin and geometry of the document
\usepackage[top=\topMargin, bottom=\bottomMargin, 
	left=\leftMargin, right=\rightMargin]{geometry} 
% package enabling calculations
\usepackage{calc}
% package changing appendix from "A Name" to "Appendix A Name"
\usepackage[titletoc,title]{appendix}

% tbh not sure why this is here or what it does...
% Haven't used it in any other LaTeX documents before, but oh well :-P
%\usepackage{array} 

% package to nicely format URLs and hyperlinks
\usepackage{hyperref} % \url{...}
\hypersetup{colorlinks, urlcolor=blue}

% ------------- COLOURS --------------%
% allows for coloured tables
\usepackage{color, colortbl} 
% allows for coloured headings
\usepackage{sectsty} 

% sets a colour
\definecolor{sectionColour}{RGB}{\sectionColour} 
\definecolor{subSectionColour}{RGB}{\subsectionColour}
\definecolor{subSubSectionColour}{RGB}{\subsubsectionColour}
% used for table heading rows
\definecolor{tableHeadColour}{RGB}{\tableHeadColour} 

% sets section heading colour
\sectionfont{\color{sectionColour}} 
% sets subsection heading colour
\subsectionfont{\color{subSectionColour}} 
% sets subsubsection heading colour
\subsubsectionfont{\color{subSubSectionColour}} 

% ------ ADDITIONAL STYLING ------%
% minipage styling (not sure if used, but might be)
\usepackage{genmpage}
% allows for list and table specification, including merged cells
\usepackage{enumitem, multicol, multirow} 
% allows for multi-page tables, as well as tables as wide as you want
\usepackage{longtable} 

% removes indent at beginning of all paragraph
\setlength\parindent{0pt} 
% removes line spacing between list items.
\setlist{noitemsep} 

% ------------ PAGE SIZES ------------ %
% Long side of A4 paper
\deflength{\aflong}{29.7cm}
% Short side of A4 paper
\deflength{\afshort}{21cm}
% Long side of A3 paper
\deflength{\atlong}{42cm}
% Short side of A3 paper
\deflength{\atshort}{29.7cm}

% Text width can be determined using (pagewidth - 2\totalmargin)

% Set page size with:
% \eject \pdfpagewidth=<width> \pdfpageheight=<height>

% Set a page to a new size (\setpagesize{width}{height})
\newcommand{\setpagesize}[2]{
		\eject 
		% set height and width
		\pdfpagewidth = #1 \pdfpageheight = #2
		% adjust horizontal and vertical offset w.r.t. a4 portrait basis
		\newgeometry{
			top = \topMargin, 
			bottom = \bottomMargin - (#2 - \aflong),
			left = \leftMargin,
			right = \rightMargin - (#1 - \afshort)
		}
		% refresh header and footer to suit new page size
		\fancyheadoffset[R]{0cm}  \fancyfootoffset[R]{0cm}
}

% Sets page to A4 portrait
\newcommand{\setafport}{\setpagesize{\afshort}{\aflong}}

% Sets page to A3 landscape
\newcommand{\setatland}{\setpagesize{\atlong}{\atshort}}

% Sets page to A4 landscape
\newcommand{\setafland}{\setpagesize{\aflong}{\afshort}}

% Sets page to A3 portrait
\newcommand{\setatport}{\setpagesize{\atshort}{\atlong}} 

% an 'owner' command for specifying who owns a given topic
\newcommand{\owner}[1]{\textit{(#1)}}

% a 'topic' command for bolded topics in the agenda and minutes
% \topic[owner]{topic}
\newcommand{\topic}[2][]{  
  \item \textbf{#2}
  \ifx&#1&%
    % do nothing
  \else
    \owner{#1}
  \fi
}

% an allocation command for specifying who needs to do what
\newcommand{\alloc}[1]{\textbf{#1:}}  % \alloc{Name} things to do


%% FIGURES/IMAGES
% graphics paths, relative to current folder.
\graphicspath{{images/}}






%% DOCUMENT BEGINS HERE

\begin{document}
\setafport
\pagenumbering{arabic} %sets body page numbering to arabic numerals

\meetingInfo % start off with location, time, and people present

\sectionl{Proposed Agenda}

	\begin{enumerate}
		% \topic[Owner(s)]{Topic Name/Descriptor}
		\topic[First2]{Topic1}\\
			Extra details
		\topic[First1, First2]{Topic2}
		\topic{Topic3} % no owner implies relevant to and owned by all
	\end{enumerate}
    		
\sectionl{Meeting Minutes}

	\begin{enumerate}
		% \topic{Topic Name/Descriptor}
		\topic{Topic1}\\
			Relevant notes/decisions
		\topic{Topic2}\\
			Removed from agenda.
		\topic{Topic3}\\
		  New client communications and plan established.
	\end{enumerate}
	
\sectionl{Next Steps}

	\begin{itemize}
	  \topic{Next Meeting}\\
			% Date, Time, Location (as accurately as known)
			Two weeks (26/2 or 28/2)
		\topic{TODO}\\
			% Actions to complete (with due date (no due date implies next meeting)
			%   and allocation to relevant person/people)
			\alloc{First1} prepare X resources.\\
			\alloc{First1, First2} make a document template (14/2), run tests.
	\end{itemize}


\end{document}






%%%%%%%%%%%%%%%%%%%%%%%%%%%%%%%%%%%%
% -------------------- COMMAND USAGE ---------------------- %

\comment{%%
    %% MINIPAGES
    \begin{minipage}{0.4\linewidth}
    % minipages have their own size within the main page, as well as their
    % own footnotes and other specified geometry/text features.
    \end{minipage}
    
    %% LISTS
    % removes space before list
    \vspace{-0.7\topsep} 
    \begin{itemize}
        \item Main
        \begin{itemize}
            \item[o] Sub
        \end{itemize}
    \end{itemize}
    
    %% FOOTNOTES
    % Sets the footnotetext with the key of the contents of the square
    % brackets (used when creating footnotes in minipages)
    \footnotetext[1]{Text in footnote} 
    % in the main page  standard footnotes can be created using
    % \footnote{•}
    \footnote{Text in footnote}
    
    %% PAGENUMBERING
    \pagenumbering{roman} % Roman numerals (contents, cover page)
    \pagenumbering{arabic} % Arabic numerals (body, appendices)
    
    %% COLUMNS
    % Sets two columns for this section of the document
    \begin{multicols}{2} 
        <Text in first column>
        \columnbreak % this breaks to the next column
        <Text in second column>
    \end{multicols}
    
    %% TABLES
    \begin{table}[H]
        \footnotesize % sets text size to that of footnotes, within table
        \centering % centers table on screen
        \caption{Table Caption}
        \label{tab: Table Caption}
        % Using p{n\linewidth} sets that column to justified, with width n times
        % text-width of the line.
        % Using c sets that column to be centered, with width auto-scaled.
        % Vertical line sets that column to have a vertical border there
        % Space separates columns without a border.
        \begin{tabular}{|c|c|p{0.5\linewidth}|} 
            \hline % adds a horizontal line
            \rowcolor{tableHeadColour} 
            		\textbf{Heading1}  & \textbf{Heading2} &
            		\textbf{Heading3}  \\ %\\ changes to next row of table
            \hline
            \textbf{Side Heading} & Text & Text\\
            \hline
        \end{tabular}
    \end{table}
    
    %% CODE
    % the following code allows for displaying programming code, telling
    % LaTeX to just print the contents rather than try to read it as LaTeX
    % text.
    \begin{lstlisting}
        /** Code goes in here */
    \end{lstlisting}
    
    %% PAGE RESIZING/ROTATING
    \setafport % Make the following pages A4 portrait
    \setafland % Make the following pages A4 landscape
    \setatport % Make the following pages A3 landscape
    \setatland % Make the following pages A3 landscape
    
    %% FIGURES
    % Insert figures with 
    \fig{scale}{filename}{caption \label{fig: <label>}}
    % Insert framed figures with
    \framefig{scale}{filename}{caption \label{fig: <label>}}
    % Insert sub-figures (within a figure)
	\subfig{alignment}{fig-width}{img-scale/height/width}{filename}{caption}
	% alignment = t(top) / b(bottom)
	% width e.g. = 0.5\textwidth
	% scale e.g. = scale=0.5 / height=5cm / width=3in
	% filename e.g. = image / image.jpg / etc
	% caption e.g. = Descriptive subfigure caption
	%% e.g.
	\begin{figure}[H]
		\centering
		\subfig{t}{}{}{}{caption \label{sfig: caption}}
		~ % (next column)
		\subfig{t}{}{}{}{caption \label{sfig: caption}}
		\caption{caption \label{fig: caption}}
	\end{figure}
	
	%% CODE
	\matlab{filename}{label} % inputs a formatted .m file with caption "Listing: label"
	\python{filename}{label} % inputs a formatted .py file with caption "Listing: label"
	 
	 %% REFERENCING
	 % Reference figures:
	 \figref{label}
	 % Reference subfigures:
	  \sfigref{label}
	 % Reference tables:
	 \tabref{label}
	 % Reference code:
	 \coderef{label}
	 % Reference sections:
	 \secref{label}
    
    %% MULTI-LINE COMMENTS
	\comment{<text>} % multiple lines of comment in the tex code
	\dcom{<text>} % multi-line comment displayed in red in the document
	
	%% UNITS IN MATH ENVIRONMENT
	% insert units into the math environment, with a space included
	% between the maths and the unit (e.g. \unit{m}^3)
	\unit{<unit>}
	% insert units without a space
	\nsunit{<unit>}
	
	%% POWERS OF TEN
	% Insert *10^n with \E{n} (e.g. \E{-5})
	\E{power} % from within math environment
	\tE{power} % from within text environment
	
	%% ADDITIONAL MATH ENVIRONMENT SHORTCUTS
	% Math formatted differentials (e.g. \D x)
	\D var
	
	% Bolded vector notation (e.g. \ve{r})
	\ve{var}
	
	% Commands for left/right surrounding braces 
	\lr{thing} % -> \left( thing \right), 
	\lrs{thing} %-> \left[ thing \right],
	\lra{thing} %-> \left | thing \right |

	% Commands for integrals
	% 		\eint{integrand}{wrt var}
	%		\eiint{integrand}{wrt var inner}{wrt var outer}
	%		\dint{lower limit}{upper limit}{integrand}{wrt var}
	%		\diint{outer_low}{out_high}{inner_low}{in_high}{integrand}{in_var}{out_var}
	
	% Partial Derivatives (e.g. \del{x}{y})
	\del{func/var}{wrt var}
	
	% Full Derivatives - nth order (e.g. \der[2]{y}{t} or \der{y}{x}
	\der[order]{func/var}{wrt var} % order is optional, hence square brackets
	
	% Trigonometric (optional power) (e.g. \cos[2]{x} or \sin{4} or \tan[-1]{y})
	\cos[order]{argument}
	\sin[order]{argument}
	\tan[order]{argument}	
	
	% Logarithm (optional base) (e.g. \log[2]{x} or \log{y})
	\log[base]{argument}

	% Quadratic Formula (e.g. \qf{5}{3}{2})
	\qf{a}{b}{c}
	
	% 2x2 Eigenvalues working (e.g. \evals{X}{x11}{x21}{x12}{x22})
	
	% Command for 2x2 eigenvectors working 
	%		(e.g. \evecs{X}{\ve{v}}{v1}{v2}{x11}{x21}{x12}{x22})
}

%%%%%%%%%%%%%%%%%%%%%%%%%%%%%%%%%%%%
